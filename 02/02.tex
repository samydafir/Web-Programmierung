\documentclass[12pt, a4paper]{report}
\usepackage[utf8]{inputenc}
\usepackage[english]{babel}
\usepackage[backend=biber,style=numeric]{biblatex}
\usepackage{csquotes}
\usepackage{graphicx}
\usepackage{epstopdf}
\usepackage{float}
\usepackage{moreverb}

\title{Übungsblatt 02}
\author{Thomas Samy Dafir, Lex Winandy}
\date{}
\hfuzz=10.0pt

\begin{document}
\maketitle

\section*{Aufgabe 1}
\textbf{\textit{Gruppe: Konfigurieren Sie im Webserver einen VirtualHost, der auf port 8765 lauscht.
Erstellen Sie ein separates DocumentRoot für diesen VirtualHost.}}\\
Zum Konfigurieren eines VirtualHost muss man in /etc/apache2/sites-available/ eine neue .conf Datei erstellen. In dieser Datei soll nun der VirtualHost eingetragen werden und auf welchem Port er lauscht. \\
\textit{$<$VirtualHost *:8765$>$ ... $<$/VirtualHost$>$} $\Rightarrow$ Angabe auf welchen Port er reagieren soll\\
\textit{DocumentRoot /var/www/virtualHost/} $\Rightarrow$ Angabe wo das Basisverzeichnis liegt \\
\textit{sudo a2ensite virtualHost.conf} $\Rightarrow$ VirtualHost aktivieren

\section*{Aufgabe 2}
\textbf{\textit{Erstellen Sie eine passwortgeschützte URL. Verwenden Sie digest authentication. Was ist
der Unterschied zu plain. Hinweis: .htaccess, mod auth digest.}}\\

Funktion: Der Client sendet eine Anfrage an den Server. Dieser antwortet mit ''not authorized'' und sendet einen einmal
gültigen Zufallswert mit. Der Client wird zur Passworteingabe aufgefordert. Der User authentufuziert sich.
Der Client sendet einen neuen Request, jetzt aber mit einem zusätzlichen Header: WWW\_Authenticate. Enthält
User, realm, qop, den Zufallswert und das gehashte Passwort: MD5 :(. Der Server vergleicht mit den Daten im Passwortfile
und sendet entwerde 200 OK oder wiederum 401 not authorized.
Unterschied zu basic: password wird gehashed übertragen.
\\Benötigte Module: mod\_auth\_digest + mod\_authn\_file, für User/Password file.\\
Verwendung:
\begin{enumerate}
	\item Optionen für Modul spezifizieren (2 Möglichkeiten):
	\begin{itemize}
		\item Über Directory Eintrag in der Apache config
		\item Über eine .htaccess Datei im zu schützenden Verzeichnis
	\end{itemize}
	\item Datei mit ''user:realm:password\_hash'' erstellen:\\
	htdigest [ -c ] passwdfile realm username\\
	Der Pfad zu ''passwdfile'' muss dann im .htaccess file angegeben werden
\end{enumerate}
Settings:
\begin{itemize}
	\item AuthType: Authentcation Type (digest/basic)
	\item AuthName: Realm name. Muss mit realm im password-file übereinstimmen.
	\item AuthDigestAlgorithm: Hash Algorithmus (MD5)
	\item AuthDigestNonceLifetime: Zeitraum für den der aktuelle Zufallswert gültig ist (in s)
	\item AuthDigestDomain: Domains für die diese Authentifizierung gültig ist
	\item AuthDigestQop: Quality of Protection. Nur username/passwort oder + integrity check
	\item AuthUserFile: Pfad zum password file
	\item Require: Gibt Voraussetzungen für Zutritt an: all granted, alldenied, valid-user(mit pwd file), user, group.
\end{itemize}


\section*{Aufgabe 3}
\textbf{\textit{Machen Sie sich mit dem Firefox Web Developer vertraut, insb. mit der ’Network’ An-
sicht. Öffnen Sie eine grössere Seite (z.B. www.uni-salzburg.at) welche HTTP Status
Codes treten auf? Was bedeuten Sie?}}\\
Site: uni-salzburg.at\\
Statuscodes:
\begin{itemize}
	\item 200 OK: Angefragter Inhalt wird ausgeliefert.
	\item 301 Moved Permanently: Permanenter Redirect. In diesem Fall eine Weiterleitung vom Port 80 (http) auf Port 443 (https).\\
	Auf dieser Seite wird man zusätzlich noch einmal weitergeleitet nämlich von https://uni-salzburg.at/ auf https://uni-salzburg.at/index.php?id=....\\
	Erst danach wird der content ausgeliefert.
	\item 302 Moved Temporarily: Redirect auf definierte Error-Seite bei request einer nicht vorhandenen datei.
	\item 404 Not Found: Tritt nie auf. Immer redirect auf eine vordefinierte Error Seite.
\end{itemize}

\section*{Aufgabe 4}
\textbf{\textit{Wie sind HTTP 1.1 request und response prinzipiell aufgebaut? Illustrieren Sie mit einem
eigenen Beispiel!}}\\
Aufbau:
\begin{verbatimtab}
Request = Request-Line
	  headers
	  CRLF
	  message body

Request-Line = Method SP Request-URI SP HTTP-Version CRLF
Method 		 = "PUT" | "GET" | "POST" | .....
RequestURI   = Pfad zur angefragten Datei
HTTP-Version = Protokoll version

headers 	 = Accept-* | Authorization | Expect | From | Host
		| User-Agent | Max-Forwards | cache-control |....

Response = Status-Line
	   headers
	   CRLF
	   message-body

Status-Line   = HTTP-Version SP Status-Code SP Reason-Phrase CRLF
Status-Code   = "1xx", "2xx", "3xx", "4xx", "5xx"
Reason-Phrase = "OK", "Not Found", "Permanent Redirect"

headers = Accept-Ranges | Age | ETag | Location | Proxy-Authenticate
		| Retry-After | Server | Vary | WWW-Authenticate |....

\end{verbatimtab}
Quelle: https://www.w3.org/Protocols/rfc2616/rfc2616-sec5.html\\
Beispiele (apple.com):\\
Request:\\
\begin{verbatim}
Request URL:https://www.apple.com/
Request Method:GET
Accept:text/html,application/xhtml+xml,application/xml;q=0.9,
image/webp,image/apng,*/*;q=0.8
Accept-Encoding:gzip, deflate, br
Accept-Language:en-GB,en-US;q=0.9,en;q=0.8
Cache-Control:max-age=0
Connection:keep-alive
Cookie:key=value....
Host:www.apple.com
Upgrade-Insecure-Requests:1
User-Agent:Mozilla/5.0 (Windows NT 10.0; Win64; x64) AppleWebKit/537.36
(KHTML, like Gecko) Chrome/64.0.3282.186 Safari/537.36 OPR/51.0.2830.55
\end{verbatim}


Response:\\
\begin{verbatim}
Status Code:200 OK
Cache-Control:max-age=0
Connection:keep-alive
Content-Encoding:gzip
Content-Length:6434
Content-Type:text/html; charset=UTF-8
Date:Wed, 14 Mar 2018 13:02:58 GMT
Expires:Wed, 14 Mar 2018 13:02:58 GMT
Server:Apache
Vary:Accept-Encoding
X-Content-Type-Options:nosniff
X-Frame-Options:SAMEORIGIN
X-Xss-Protection:1; mode=block
\end{verbatim}
\section*{Aufgabe 5}
\textbf{\textit{Gruppe: Was bedeuten folgende Felder im HTTP request und response header? DNT,
Connection: keep-alive, ETag, Content-Length}}
\begin{itemize}
	\item DNT: Befiehlt einer Seite, den Nutzer nicht zu tracken.
	\item Connection: keep-alive: Mit dem Feld wird die Verbindung nicht sofort abgebrochen, sondern bis einer der beiden die Verbindung abbrechen will.
	\item ETag: (entity tag)Das Feld dient zur Bestimmung von Änderungen an der angeforderten Ressource und wird hauptsächlich zum Caching verwendet. Dies bedeutet dass bei der ersten Anfrage der ETag vom Server mitgeschickt wird und sollte nochmal eine Anfrage auf die selben Ressourcen geschickt werden wird der ETag vom Client geschickt und sollte es sich um zweimal den selben Wert handeln schickt der Server die Ressourcen nicht nochmal sondern weist darauf hin, dass die Ressourcen noch immer die selben sind.
	\item Content-Length: Gibt die Länge des Bodys in Bytes an.
\end{itemize}

\section*{Aufgabe 6}
\textbf{\textit{Gruppe: Wie unterscheidet sich HTTP 1.1 von HTTP/2? Wie wird ausgehandelt welches
Protokoll gesprochen wird? Zeichnen Sie den Netzwerkverkehr auf Hinweis: tcpdump +
Wireshark oder Web Developer}}

\end{document}
