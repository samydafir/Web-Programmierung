\documentclass[12pt, a4paper]{report}
\usepackage[utf8]{inputenc}
\usepackage[english]{babel}
\usepackage[backend=biber,style=numeric]{biblatex}
\usepackage{csquotes}
\usepackage{graphicx}
\usepackage{epstopdf}
\usepackage{float}
\usepackage{moreverb}

\title{Übungsblatt 02}
\author{Thomas Samy Dafir, Lex Winandy}
\date{}
\hfuzz=10.0pt

\begin{document}
\maketitle

\section*{Aufgabe 1}
\textbf{\textit{Gruppe: Konfigurieren Sie im Webserver einen VirtualHost, der auf port 8765 lauscht.
Erstellen Sie ein separates DocumentRoot für diesen VirtualHost.}}


\section*{Aufgabe 2}
\textbf{\textit{Erstellen Sie eine passwortgeschützte URL. Verwenden Sie digest authentication. Was ist
der Unterschied zu plain. Hinweis: .htaccess, mod auth digest.}}\\

Funktion: Der Client sendet eine Anfrage an den Server. Dieser antwortet mit ''not authorized'' und sendet einen einmal
gültigen Zufallswert mit. Der Client wird zur Passworteingabe aufgefordert. Der User authentufuziert sich.
Der Client sendet einen neuen Request, jetzt aber mit einem zusätzlichen Header: WWW\_Authenticate. Enthält
User, realm, qop, den Zufallswert und das gehashte Passwort: MD5 :(. Der Server vergleicht mit den Daten im Passwortfile
und sendet entwerde 200 OK oder wiederum 401 not authorized.
Unterschied zu basic: password wird gehashed übertragen.
\\Benötigte Module: mod\_auth\_digest + mod\_authn\_file, für User/Password file.\\
Verwendung:
\begin{enumerate}
	\item Optionen für Modul spezifizieren (2 Möglichkeiten):
	\begin{itemize}
		\item Über Directory Eintrag in der Apache config
		\item Über eine .htaccess Datei im zu schützenden Verzeichnis
	\end{itemize}
	\item Datei mit ''user:realm:password\_hash'' erstellen:\\
	htdigest [ -c ] passwdfile realm username\\
	Der Pfad zu ''passwdfile'' muss dann im .htaccess file angegeben werden
\end{enumerate}
Settings:
\begin{itemize}
	\item AuthType: Authentcation Type (digest/basic)
	\item AuthName: Realm name. Muss mit realm im password-file übereinstimmen. 
	\item AuthDigestAlgorithm: Hash Algorithmus (MD5)
	\item AuthDigestNonceLifetime: Zeitraum für den der aktuelle Zufallswert gültig ist (in s)
	\item AuthDigestDomain: Domains für die diese Authentifizierung gültig ist
	\item AuthDigestQop: Quality of Protection. Nur username/passwort oder + integrity check
	\item AuthUserFile: Pfad zum password file
	\item Require: Gibt Voraussetzungen für Zutritt an: all granted, alldenied, valid-user(mit pwd file), user, group.
\end{itemize}	


\section*{Aufgabe 3}
\textbf{\textit{Machen Sie sich mit dem Firefox Web Developer vertraut, insb. mit der ’Network’ An-
sicht. Öffnen Sie eine grössere Seite (z.B. www.uni-salzburg.at) welche HTTP Status
Codes treten auf? Was bedeuten Sie?}}\\
Site: uni-salzburg.at\\
Statuscodes:
\begin{itemize}
	\item 200 OK: Angefragter Inhalt wird ausgeliefert.
	\item 301 Moved Permanently: Permanenter Redirect. In diesem Fall eine Weiterleitung vom Port 80 (http) auf Port 443
	(https). Erst danach wird der content ausgeliefert.
	\item 302 Moved Temporarily: Redirect auf definierte Error-Seite bei request einer nicht vorhandenen datei.
	\item 404 Not Found: Tritt nie auf. Immer redirect auf eine vordefinierte Error Seite.
\end{itemize}

\section*{Aufgabe 4}
\textbf{\textit{Wie sind HTTP 1.1 request und response prinzipiell aufgebaut? Illustrieren Sie mit einem
eigenen Beispiel!}}

\section*{Aufgabe 5}
\textbf{\textit{Gruppe: Was bedeuten folgende Felder im HTTP request und response header? DNT,
Connection: keep-alive, ETag, Content-Length}}

\section*{Aufgabe 6}
\textbf{\textit{Gruppe: Wie unterscheidet sich HTTP 1.1 von HTTP/2? Wie wird ausgehandelt welches
Protokoll gesprochen wird? Zeichnen Sie den Netzwerkverkehr auf Hinweis: tcpdump +
Wireshark oder Web Developer}}

\end{document}
