\documentclass[12pt, a4paper]{report}
\usepackage[utf8]{inputenc}
\usepackage[english]{babel}
\usepackage[backend=biber,style=numeric]{biblatex}
\usepackage{csquotes}
\usepackage{graphicx}
\usepackage{epstopdf}
\usepackage{float}
\usepackage{moreverb}

\title{Übungsblatt 02}
\author{Thomas Samy Dafir, Lex Winandy}
\date{}
\hfuzz=10.0pt

\begin{document}
\maketitle

\section*{Aufgabe 1}
\textbf{\textit{Gruppe: Konfigurieren Sie im Webserver einen VirtualHost, der auf port 8765 lauscht.
Erstellen Sie ein separates DocumentRoot für diesen VirtualHost.}}


\section*{Aufgabe 2}
\textbf{\textit{Erstellen Sie eine passwortgeschützte URL. Verwenden Sie digest authentication. Was ist
der Unterschied zu plain. Hinweis: .htaccess, mod auth digest.}}

\section*{Aufgabe 3}
\textbf{\textit{Machen Sie sich mit dem Firefox Web Developer vertraut, insb. mit der ’Network’ An-
sicht. Öffnen Sie eine grössere Seite (z.B. www.uni-salzburg.at) welche HTTP Status
Codes treten auf? Was bedeuten Sie?}}

\section*{Aufgabe 4}
\textbf{\textit{Wie sind HTTP 1.1 request und response prinzipiell aufgebaut? Illustrieren Sie mit einem
eigenen Beispiel!}}

\section*{Aufgabe 5}
\textbf{\textit{Gruppe: Was bedeuten folgende Felder im HTTP request und response header? DNT,
Connection: keep-alive, ETag, Content-Length}}

\section*{Aufgabe 6}
\textbf{\textit{Gruppe: Wie unterscheidet sich HTTP 1.1 von HTTP/2? Wie wird ausgehandelt welches
Protokoll gesprochen wird? Zeichnen Sie den Netzwerkverkehr auf Hinweis: tcpdump +
Wireshark oder Web Developer}}

\end{document}
