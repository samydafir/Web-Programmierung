\documentclass[12pt, a4paper]{report}
\usepackage[utf8]{inputenc}
\usepackage[english]{babel}
\usepackage[backend=biber,style=numeric]{biblatex}
\usepackage{csquotes}
\usepackage{graphicx}
\usepackage{epstopdf}
\usepackage{float}
\usepackage{moreverb}
\usepackage{hyperref}


\title{Übungsblatt 02}
\author{Thomas Samy Dafir, Lex Winandy}
\date{}
\hfuzz=10.0pt

\begin{document}
\maketitle

\section*{Aufgabe 1}




\section*{Aufgabe 2}

\section*{Aufgabe 3}

Implementierung siehe Links auf persönlichen Seiten unter $buntmeise.cosy.sbg.ac.at$\\
Funktionsprinzip: Datei wird geöffnet (read reicht aus). Mittels einer Schleife wird über alle Zeilen iteriert und jede Zeile
für sich analysiert. Die IP-Addresse wird extrahiert (erster String in jeder Zeile). Wurde die Adresse bereits gefunden, wird der Eintrag im Dictionary inkrementiert, ansonsten wird 1 eingefügt (1. Vorkommen der Adresse).

\section*{Aufgabe 4}
Implementierung siehe Links auf persönlichen Seiten unter $buntmeise.cosy.sbg.ac.at$\\
Das Modul $urllib3$ wird importiert. Ein ConnectionPool wird erstellt (zwar führen wir nur einen Request durch, der Pool vereinfacht aber die Handhabung). Für das Request benötigen wir die HTTP-Method und eine URL. Nach Ausführen des Requests können der Statuscode und die Header über <resuest-object>. status bzw. <resuest-object>.headers ausgelesen werden.



\end{document}
