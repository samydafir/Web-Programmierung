\documentclass[12pt, a4paper]{report}
\usepackage[utf8]{inputenc}
\usepackage[english]{babel}
\usepackage[backend=biber,style=numeric]{biblatex}
\usepackage{csquotes}
\usepackage{graphicx}
\usepackage{epstopdf}
\usepackage{float}
\usepackage{moreverb}

\title{Übungsblatt 02}
\author{Thomas Samy Dafir, Lex Winandy}
\date{}
\hfuzz=10.0pt

\begin{document}
\maketitle

\section*{Aufgabe 1}
\textbf{\textit{Gruppe: Installieren Sie ein tool um das Apache log auszuwerten (Vorschlag: webalizer, analog oder awstats). Die Ergebnisse sollen über einen Link von Ihrer Homepage sichtbar sein.}}\\

\section*{Aufgabe 2}
\textbf{\textit{Gruppe: Konfigurieren Sie im Webserver einen internen redirect und einen redirect auf eine externe Seite. Wie gehen Sie vor? Was passiert auf HTTP-Ebene? Hinweis: mod rewrite}}\\

\section*{Aufgabe 3}
\textbf{\textit{Erstellen Sie weiters eine HTML-Seite, die den Browser anweist, auf eine andere Seite zu gehen (Hinweis: meta http-equiv="Refresh"). Was passiert auf HTTP-Ebene?}}\\
Folgendes meta-tag definiert eine Weiterleitung.
\begin{verbatim}
<meta http-equiv="refresh" content="timeToRedirect; URL=redirectDestination">
\end{verbatim}
Diese Art der Weiterleitung ist im Header nicht sichtbar. Erst wird das erstellte Dokument geladen, das den redirect erhält: 200 OK. Nach der angegebenen Zeit
erfolgt die Weiterleitung auf die angegebene Seite. Das entspricht einer clientseitigen Anforderung und ist im Header nicht als Redirect zu erkennen. Man erhält
den Status 200 OK (falls das Dokument existiert und der Server erreichbar ist natürlich).


\section*{Aufgabe 4}
\textbf{\textit{Zeichnen Sie das HTTP-Protokoll auf wenn Sie sich zu http://pmeerw.net/www-mm18/geheim verbinden. Username und Passwort sind geheim, geheim. Welche HTTP Requests bzw. Responses und welche Status Codes treten auf und was bedeuten sie? Wann muss das Passwort erneut eingegeben werden?}}\\

\section*{Aufgabe 5}
\textbf{\textit{Erstellen Sie eine ’persönliche’ Seite in HTML5 fur sich (Name, Adresse, Kontaktmöglichkeit, Beschreibung, Bild, Lebenslauf, etc.). Achten Sie dabei ausschließlich auf die Struktur der Seite, nicht die optische Gestaltung (kein CSS).}}

\begin{itemize}
  \item[a] Was ist die Bedeutung von des Attributs alt bei $<img>$?
  \item[b] Welche Meta-Information sollte in den Header damit die Seite (leichter) gefunden wird?
  \item[c] Welche semantischen HTML-Tags können verwendet werden um die Struktur zu kodieren?
\end{itemize}

\end{document}
