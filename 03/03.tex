\documentclass[12pt, a4paper]{report}
\usepackage[utf8]{inputenc}
\usepackage[english]{babel}
\usepackage[backend=biber,style=numeric]{biblatex}
\usepackage{csquotes}
\usepackage{graphicx}
\usepackage{epstopdf}
\usepackage{float}
\usepackage{moreverb}
\usepackage{hyperref}


\title{Übungsblatt 02}
\author{Thomas Samy Dafir, Lex Winandy}
\date{}
\hfuzz=10.0pt

\begin{document}
\maketitle

\section*{Aufgabe 1}
\textbf{\textit{Gruppe: Installieren Sie ein tool um das Apache log auszuwerten (Vorschlag: webalizer, analog oder awstats). Die Ergebnisse sollen über einen Link von Ihrer Homepage sichtbar sein.}}\\
Wir verwenden 'Webalizer'\\
Schritte, um webalizer für unsere Seite zu verwenden:\\
\begin{enumerate}
	\item sudo apt-get install webalizer:\\
	Richtet webalizer für /var/www/webalizer ein.
	\item sudo mv /var/www/webalizer /var/www/html/
	\item Webalizer konfigurieren:\\
	sudo nano /etc/webalizer/webalizer.conf
	\item Settings:Logfile angeben:
	LogFile /var/log/apache2/access.log\\
	Output directory auf ordner in unserem web root directory $\rightarrow$ html Output und Anzeige:\\
	OutputDir /var/www/html/webalizer
	\item Konfig testen (webalizer starten):\\
	sudo webalizer
\end{enumerate}


\section*{Aufgabe 2}
\textbf{\textit{Gruppe: Konfigurieren Sie im Webserver einen internen redirect und einen redirect auf eine externe Seite. Wie gehen Sie vor? Was passiert auf HTTP-Ebene? Hinweis: mod rewrite}}\\
\begin{itemize}
	\item sudo a2enmod rewrite
	\item sudo apache2 restart
	\item In der .htaccess Datei
	\begin{itemize}
		\item RewriteEngine on
		\item RewriteRule pattern substitution [flags]
	\end{itemize}
\end{itemize}

\section*{Aufgabe 3}
\textbf{\textit{Erstellen Sie weiters eine HTML-Seite, die den Browser anweist, auf eine andere Seite zu gehen (Hinweis: meta http-equiv="Refresh"). Was passiert auf HTTP-Ebene?}}\\
Folgendes meta-tag definiert eine Weiterleitung.
\begin{verbatim}
<meta http-equiv="refresh" content="timeToRedirect; URL=redirectDestination">
\end{verbatim}
Diese Art der Weiterleitung ist im Header nicht sichtbar. Erst wird das erstellte Dokument geladen, das den redirect erhält: 200 OK. Nach der angegebenen Zeit
erfolgt die Weiterleitung auf die angegebene Seite. Das entspricht einer clientseitigen Anforderung und ist im Header nicht als Redirect zu erkennen. Man erhält
den Status 200 OK (falls das Dokument existiert und der Server erreichbar ist natürlich).


\section*{Aufgabe 4}
\textbf{\textit{Zeichnen Sie das HTTP-Protokoll auf wenn Sie sich zu http://pmeerw.net/www-mm18/geheim verbinden. Username und Passwort sind geheim, geheim. Welche HTTP Requests bzw. Responses und welche Status Codes treten auf und was bedeuten sie? Wann muss das Passwort erneut eingegeben werden?}}\\
Es wird das HTTP/1.1 benutt. Nach dem GET request bekommen wir den Status Code 302: Found (Umleitung). Das Passwort muss nicht nochmal angegeben werden solange man den Cache seines Browsers nicht löscht.

\section*{Aufgabe 5}
\textbf{\textit{Erstellen Sie eine ’persönliche’ Seite in HTML5 für sich (Name, Adresse, Kontaktmöglichkeit, Beschreibung, Bild, Lebenslauf, etc.). Achten Sie dabei ausschließlich auf die Struktur der Seite, nicht die optische Gestaltung (kein CSS).}}

\begin{itemize}
  \item[a] Was ist die Bedeutung des Attributs alt bei $<img>$?
  'alt' für alternate gibt einen alternative Text an, der angezeigt wird, falls das Bild nicht
  angezeigt werden kann. 'alt' trägt auch zur Accessibility der Seite bei, das es z.B. von Screenreadern
  gelesen wird.
  \item[b] Welche Meta-Information sollte in den Header damit die Seite (leichter) gefunden wird?
  Von der Überflutung mit meta-tags wird heute abgeraten. Viel wichtiger für eine gute google-Indizierung ist
  heute das Gesamtbild der Website:
  \begin{itemize}
  	\item Aussagekräftiger Titel
  	\item Struktur der Seite selbst: Sections, Links, Bilder,... aussagekräftig benennen.
  	\item 'description' meta Tag mit guter Beschreibung verwenden.
  	\item Accessibility: z.B. 'lang' Attribut im html-tag $\rightarrow$ für Screenreader.
  	\item Mobile-friendly
  	\item Korrekter Code
  	\item Keine externen Weiterleitungen/Links auf die eigene Seite platzieren (don't try to fool the algorithm).
  \end{itemize}
  \item[c] Welche semantischen HTML-Tags können verwendet werden um die Struktur zu kodieren?\\
  Semantische Tags sind dazu dar, Content inhaltlich zu trennen. Sie trennen sich nur manchmal auf die
  Formatierung aus.
  Vollständige Liste unter:\\
  https://www.w3schools.com/html/html5\_semantic\_elements.asp\\
  Verwendete Elemente:\\
  \begin{itemize}
  	\item section
  	\item header
  	\item footer
  	\item details
  	\item summary
  	\item time
  \end{itemize}
\end{itemize}

\end{document}
