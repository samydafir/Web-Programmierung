\documentclass[12pt, a4paper]{report}
\usepackage[utf8]{inputenc}
\usepackage[english]{babel}
\usepackage[backend=biber,style=numeric]{biblatex}
\usepackage{csquotes}
\usepackage{graphicx}
\usepackage{epstopdf}
\usepackage{float}
\usepackage{moreverb}
\usepackage{hyperref}
\setlength{\parindent}{0em} 

\title{Übungsblatt 07}
\author{Thomas Samy Dafir, Lex Winandy}
\date{}
\hfuzz=10.0pt

\begin{document}
\maketitle

Links zu allen Lösungen sind unter $buntmeise.sbg.ac.at$ aufrufbar.

\section*{Aufgabe 1}
Funktionsweise:\\
\begin{itemize}
	\item Formular implementiert mit Monats- und Jahreseingabe.
	\item Dieses Formular wird an 'b07\_a01.cgi' gesendet
	\item Daten wierden mittels 'FieldStorage' extrahiert.
	\item Mit 'subprocess.Popen' rufen wir 'cal' mit Monat und Jahr als Argumente auf und 'pipen' das Resultat, um
	es dann auszulesen und in einer Variable zu speichern.
	\item Der Kalender-String wird dann von 'pre' Tags umgeben und ausgegeben
\end{itemize}

\section*{Aufgabe 2}
Funktionsweise:\\
\begin{itemize}
	\item Öffnen der Datei 'counter.txt' im 'r+' Modus, der lesen, swie schreiben erlaubt.
	\item Um inkrementieren und zurücksetzen zu ermöglichen, implementieren wir Buttons:\\
	Ein button inkrementiert, der andere setzt den Zähler zurück.
	\item Auf die Formulardaten greifen wir mit 'cgi.FieldStorage' zu.
	\item FieldStorage wird auf den 'action' key geprüft, um zu checken, ob ein Button gedrückt wurde.
	\item Wurde kein Button gedrückt, wird der aktuelle Wert in 'counter.txt' zurückgegeben.
	\item Wurde 'reset' geklickt, wird der counter zurückgesetzt (das File geleert und 1 geschrieben).
	\item Wurde 'increment' geklickt, wird der Wert im File gelesen, das File geleert und der inkrementierte
	Wert geschrieben.
\end{itemize}

Gleichzeitiger Zugriff:
Implementiert durch ein einfaches cmd script, dass einen Browser öffnet und die 'increment Seite' mehrfach gleichzeitig aufruft.

\section*{Aufgabe 3}
Verwendete Module: os, cgi, Cookie\\
Funktionsweise:\\
\begin{itemize}
	\item Ein Cookie und ein FieldStorage Objekt werden erstellt.
	\item Ist das 'visited' Cookie nicht vorhanden, wird es erstellt und als Header ausgegeben.
	\item ist das Cookie bereits gesetzt, erfolgt keine Aktion. Status wird ausgegeben.
	\item Ist das Cookie gesetzt und Löschen wurde angefordert, wird das Cookie gelöscht: Eintrag im Cookie dictionary wird gelöscht und 'expires' auf ein vergangenes Datum gesetzt. Der mitgesendete Set-Cookie Header weist den Browser an, das Cookie zu löschen.
\end{itemize}


\section*{Aufgabe 4}
Module: cgi, Cookie, os, cPickle\\
Funktionsweise:\\
\begin{itemize}
	\item FieldStorage and Cookie Objekt werden erstellt
	\item Mit os.environ wird überprüft, ob ein Cookie gesetzt wird. Falls ja, wird dieses ins Cookie Objekt geladen.
	\item Es wird überprüft, ob 'colour\_id' im Cookie existiert. Falls ja wird pickle geladen und der zur id gehörende Farbwert ausgelesen
	\item Wurde zusätzlich eine Farbe mit dem Request übermittelt und ist diese definiert, wird die ID gelesen, umgesetzt, als neue Farbe für die Anzeige definiert und im Cookie gespeichert.
	\item Wurde noch kein Cookie gesetzt, wird das dictionary für pickle initialisiert und gespeichert, sowie das Cookie gesetzt.
	\item Die Farbe wird durch einen inline css style im body Tag gesetzt.
\end{itemize}



\end{document}
