\documentclass[12pt, a4paper]{report}
\usepackage[utf8]{inputenc}
\usepackage[english]{babel}
\usepackage[backend=biber,style=numeric]{biblatex}
\usepackage{csquotes}
\usepackage{graphicx}
\usepackage{epstopdf}
\usepackage{float}
\usepackage{moreverb}
\usepackage{hyperref}


\title{Übungsblatt 02}
\author{Thomas Samy Dafir, Lex Winandy}
\date{}
\hfuzz=10.0pt

\begin{document}
\maketitle

\section*{Aufgabe 1}
\textbf{\textit{Gruppe: Erweitern Sie Ihren Apachen so, dass auch HTTPS unterstützt wird. Erstellen
Sie dazu ein self-signed certificate. Notieren Sie die Schritte die Sie durchgeführt haben,
bzw. erklären Sie die Schritte die Sie in einem Tutorial gefunden haben. HTTPS mit dem self-signed certificate soll auf Port 
8443 angeboten werden. Zeigen Sie im Browser, dass ihr Zertifikat funktioniert.}}\\
\begin{enumerate}
	\item SSL Modul aktivieren:
	\begin{verbatim}
sudo a2enmod ssl
	\end{verbatim}
	\item Directory für Keys erstellen und Berechtigungen setzen:
	\begin{verbatim}
sudo mkdir /etc/apache2/ssl
sudo chmod -r 600 /etc/apache2/ssl
	\end{verbatim}
	\item Key und Zertifikat erstellen:
	\begin{verbatim}
sudo openssl req -x509 -nodes -days 1095 -newkey rsa:2048 
-keyout /etc/apache2/ssl/self_signed.key 
-out /etc/apache2/ssl/self_signed.crt
	\end{verbatim}
	\item info ausfüllen: WICHTIG: Common name / server name / IP address
	\item Default SSL site kopieren und konfigurieren:
	\begin{verbatim}
<VirtualHost _default_:8443>	
ServerName buntmeise.cosy.sbg.ac.at
DocumentRoot /etc/apache2/html
SSLCertificateFile /etc/apache2/ssl/self_signed.crt
SSLCertificateKeyFile /etc/apache2/ssl/self_signed.key
	\end{verbatim}
\item Port in Apache setzen:
	\begin{verbatim}
<IfModule ssl_module>
	Listen 8443
</IfModule>
	\end{verbatim}
\item Site aktivieren:
	\begin{verbatim}
sudo a2ensite neue_ssl_site.conf
	\end{verbatim}	
	\item Apache neu starten:
	\begin{verbatim}
sudo service apache2 restart
	\end{verbatim}
\item Die Seite ist jetzt erreichbar unter:\\
https://buntmeise.cosy.sbg.ac.at:8443
\end{enumerate}
Quelle; https://www.digitalocean.com/community/tutorials/how-to-create-a-ssl-certificate-on-apache-for-ubuntu-14-04

\section*{Aufgabe 2}
\textbf{\textit{Gruppe: Installieren Sie ein SSL Zertifikation von https://letsencrypt.org/ für die
Verwendung auf Port 443. Was ist der Unterschied zu einem self-signed Zertifikat?}}\\
\begin{enumerate}
	\item Certbot installieren:
	\begin{verbatim}
sudo apt-get update
sudo apt-get install software-properties-common
sudo add-apt-repository ppa:certbot/certbot
sudo apt-get update
sudo apt-get install python-certbot-apache 
	\end{verbatim}
	\item Certbot ausführen. $certonly$ verwendet, damit certbot die apache config nicht verändert. Zertifikat wird unter $/etc/letsencrypt$ erstellt.
	\begin{verbatim}
sudo certbot --apache certonly
	\end{verbatim}
	\item Default SSL site kopieren und konfigurieren:
	\begin{verbatim}
<VirtualHost _default_:443>	
ServerName buntmeise.cosy.sbg.ac.at
DocumentRoot /etc/apache2/html
Include /etc/letsencrypt/options-ssl-apache.conf
SSLCertificateFile /etc/letsencrypt/live/buntmeise.cosy.sbg.ac.at/
fullchain.pem
SSLCertificateKeyFile /etc/letsencrypt/live/buntmeise.cosy.sbg.ac.at/
privkey.pem
	\end{verbatim}
	\item Port in Apache setzen:
	\begin{verbatim}
<IfModule ssl_module>
Listen 443
</IfModule>
	\end{verbatim}
	\item Site aktivieren:
	\begin{verbatim}
sudo a2ensite letsencrypt_site.conf
	\end{verbatim}	
	\item Apache neu starten:
	\begin{verbatim}
sudo service apache2 restart
	\end{verbatim}
	\item Die Seite ist jetzt erreichbar unter:\\
	https://buntmeise.cosy.sbg.ac.at:443
	\item Zusätzlich setzen wir folgende Einstellungen, um die Verschlüsselungsqualität zu maximieren. Wir erlauben nur TLSv1.2 und starke Cipher Algorithmen:
	\begin{verbatim}
SSLProtocol         all -SSLv3 -TLSv1 -TLSv1.1
SSLCipherSuite      ECDHE-ECDSA-AES256-GCM-SHA384:
ECDHE-RSA-AES256-GCM-SHA384:ECDHE-ECDSA-CHACHA20-POLY1305:
ECDHE-RSA-CHACHA20-POLY1305:ECDHE-ECDSA-AES128-GCM-SHA256:
ECDHE-RSA-AES128-GCM-SHA256:ECDHE-ECDSA-AES256-SHA384:
ECDHE-RSA-AES256-SHA384:ECDHE-ECDSA-AES128-SHA256:
ECDHE-RSA-AES128-SHA256
SSLHonorCipherOrder on
SSLCompression      off
SSLSessionTickets   off
	\end{verbatim}
Ein self-signed Zertifikat ermöglicht nur die Verschlüsselung der Verbindung, bestätigt jedoch nicht die Identität des Servers, da es von keiner Certificate Authority signiert ist. Das führt dazu, dass Browser beim Besuch der Seite warnen (kein grünes Schloss). Das Zertifikat von letsencrypt ist jedoch von einer CA signiert $\rightarrow$ Identität bestätigt. 
\end{enumerate}
Quellen:\\
https://certbot.eff.org/lets-encrypt/ubuntuxenial-apache\\
https://www.digitalocean.com/community/tutorials/how-to-secure-apache-with-let-s-encrypt-on-ubuntu-16-04\\
https://httpd.apache.org/docs/trunk/ssl/ssl\_howto.html


\section*{Aufgabe 3}
\textbf{\textit{Gruppe: Spielen Sie mit https://www.ssllabs.com/ssltest/. Welche Grade erreicht
die e-commerce site Ihres Vertrauens? Welche Sicherheit erreicht Ihre Meise? Testen Sie
mit dem Skript testssl.sh von http://testssl.sh/ Ihren Server!}}

\section*{Aufgabe 4}
\textbf{\textit{Jeder: Erstellen Sie eine HTML-Seite ohne CSS die in etwa so aussieht wie unten angegeben.
Überprüfen Sie das Ergebnis mit dem Validator. Vergleichen Sie das Ergebnis
mit zumindest zwei Browsern (Screenshot!)}}\\
Ergebnisse: siehe persönliche Seiten auf $buntmeise.cosy.sbg.ac.at$\\
Das generelle Layout unterscheidet sich nicht. Die Darstellung in verschiedenen Browsern unterscheidet sich aufgrund deren
unterschiedlicher Default-Styles. Diese sind heute aber schon sehr gut aneinander angepasst und daher die Darstellungen sehr ähnlich. In unserem Beispiel unterscheidet sich die Darstellung verschiedener Abschnitte minimal voneinander. Unterschiede sind z.B. Abstände zwischen Buchstaben und Wörtern, Standardgröße von Text in verschiedenen Elementen (sub, sup, code,...), Darstelung einiger Zeichen (\&), sowie Absatzabstände.\\
Getestete Browser: Opera, Chrome, Firefox, Edge, Internet Explorer



\section*{Aufgabe 5}
\textbf{\textit{Jeder: Erstellen Sie ein Such-Formular, das Anfragen an http://google.com/search
per GET request richtet (q=). Ihr Formular soll die Anzahl der Ergebnisse einschränken
können (num=, per Dropdown-Auswahl).
Lesen Sie: http://moz.com/ugc/the-ultimate-guide-to-the-google-search-parameters}}\\
Links zu Formularen auf den persönlichen Seiten unter $buntmeise.cosy.sbg.ac.at$

\section*{Aufgabe 6}
\textbf{\textit{Jeder: Was ist der Unterschied zwischen HTTP GET und POST? Was sind die Vor- und
Nachteile? Recherchieren Sie! Zeichnen Sie die jeweiligen requests auf und analysieren
Sie die Unterschiede!}}








\end{document}
