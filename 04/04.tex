\documentclass[12pt, a4paper]{report}
\usepackage[utf8]{inputenc}
\usepackage[english]{babel}
\usepackage[backend=biber,style=numeric]{biblatex}
\usepackage{csquotes}
\usepackage{graphicx}
\usepackage{epstopdf}
\usepackage{float}
\usepackage{moreverb}
\usepackage{hyperref}


\title{Übungsblatt 02}
\author{Thomas Samy Dafir, Lex Winandy}
\date{}
\hfuzz=10.0pt

\begin{document}
\maketitle

\section*{Aufgabe 1}
\textbf{\textit{Gruppe: Erweitern Sie Ihren Apachen so, dass auch HTTPS unterstützt wird. Erstellen
Sie dazu ein self-signed certificate. Notieren Sie die Schritte die Sie durchgeführt haben,
bzw. erklären Sie die Schritte die Sie in einem Tutorial gefunden haben. HTTPS mit dem self-signed certificate soll auf Port 
8443 angeboten werden. Zeigen Sie im Browser, dass ihr Zertifikat funktioniert.}}\\
\begin{enumerate}
	\item 
	\begin{verbatim}
sudo openssl req -x509 -nodes -days 1095 -newkey rsa:2048 
-keyout /etc/apache2/ssl/self_signed.key 
-out /etc/apache2/ssl/self_signed.crt
	\end{verbatim}
	\item info ausfüllen: WICHTIG: Common name / server name / IP address
	\item 
\end{enumerate}


\section*{Aufgabe 2}
\textbf{\textit{Gruppe: Installieren Sie ein SSL Zertifikation von https://letsencrypt.org/ für die
Verwendung auf Port 443. Was ist der Unterschied zu einem self-signed Zertifikat?}}

\section*{Aufgabe 3}
\textbf{\textit{Gruppe: Spielen Sie mit https://www.ssllabs.com/ssltest/. Welche Grade erreicht
die e-commerce site Ihres Vertrauens? Welche Sicherheit erreicht Ihre Meise? Testen Sie
mit dem Skript testssl.sh von http://testssl.sh/ Ihren Server!}}

\section*{Aufgabe 4}
\textbf{\textit{Jeder: Erstellen Sie eine HTML-Seite ohne CSS die in etwa so aussieht wie unten angegeben.
Überprüfen Sie das Ergebnis mit dem Validator. Vergleichen Sie das Ergebnis
mit zumindest zwei Browsern (Screenshot!)}}

\section*{Aufgabe 5}
\textbf{\textit{Jeder: Erstellen Sie ein Such-Formular, das Anfragen an http://google.com/search
per GET request richtet (q=). Ihr Formular soll die Anzahl der Ergebnisse einschränken
können (num=, per Dropdown-Auswahl).
Lesen Sie: http://moz.com/ugc/the-ultimate-guide-to-the-google-search-parameters}}

\section*{Aufgabe 6}
\textbf{\textit{Jeder: Was ist der Unterschied zwischen HTTP GET und POST? Was sind die Vor- und
Nachteile? Recherchieren Sie! Zeichnen Sie die jeweiligen requests auf und analysieren
Sie die Unterschiede!}}

\end{document}
