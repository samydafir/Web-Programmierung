\documentclass[12pt, a4paper]{report}
\usepackage[utf8]{inputenc}
\usepackage[english]{babel}
\usepackage[backend=biber,style=numeric]{biblatex}
\usepackage{csquotes}
\usepackage{graphicx}
\usepackage{epstopdf}
\usepackage{float}
\usepackage{moreverb}

\title{Übungsblatt 01}
\author{Thomas Samy Dafir, Lex Winandy}
\date{}
\hfuzz=10.0pt

\begin{document}
\maketitle

\section*{Aufgabe 1}
\textbf{\textit{Gruppe: Richten Sie Benutzer-Accounts für alle Gruppenmitglieder auf ihrem server ein. Überlegen Sie wer alles sudo Rechte bekommen soll (alle?, einer?).}}

Zum hinzufügen von Benutzer-Accounts benutzt man den Befehl "sudo adduser \textit{username}". Um ihm dann noch sudo Rechte zu geben muss man ihn zur Sudo Gruppe hinzufügen: "sudo adduser \textit{username} sudo"
\section*{Aufgabe 2}
\textbf{\textit{Gruppe: Installieren Sie den Webserver Apache auf Ihrem server. Erstellen Sie eine passende Startseite in HTML (/var/www/.../ index.html). Diese soll zumindest den Rechnernamen, ein Bild ihres Vogels, und Links auf die Homepages der Gruppenmitglieder enthalten.}}
\begin{itemize}
	\item "sudo apt-get install apache2" $\Rightarrow$ Apache installiert
	\item $<$a href="\textit{url}"$>$ $\Rightarrow$ Link eingefügt
	\item $<$img src="buntmeise.JPG" alt="buntmeise"$>$ $\Rightarrow$ Bild eingefügt
\end{itemize}

\section*{Aufgabe 3}
\textbf{\textit{Gruppe: Konfigurieren Sie Apache so, dass jeder Benutzer auf ihrem server eine Homepage einrichten kann. Diese soll erreichbar sein unter http://vogel.cosy.sbg.ac.at/~benutzer/.
Hinweis: ’apache per user web directories ubuntu’}}

\begin{itemize}
	\item Benötigt Apache Modul: userdir
	\item Aktivieren: sudo a2enmod userdir
	\item Root directory festlegen: UserDir public\_html.
	\item Directory ~/public\_html anlegen
	\item Testen: buntmeise.cosy.sbg.ac.at/~username
\end{itemize}

\section*{Aufgabe 4}
\textbf{\textit{Erklären Sie Anforderungen an ein sicheres Passwort. Wie haben Sie ihr Passwort gewählt?
(ohne dabei Ihr Passwort zu verraten :) Hinweis: ’xkcd’ und ’password’}}

\begin{itemize}
	\item Länge (mindesens 8 Zeichen, Brute Force)
	\item Keine Wörter (Dictionary Attacks)
	\item Am Besten zufällig aus großen Character-Set generiert
\end{itemize}

\section*{Aufgabe 5}
\textbf{\textit{Erstellen Sie eine HTML-Seite die Umlaute (Äüöß) und Zeichen wie $\&, <, >$ enthält. Welche character encodings gibt es? Wie kann das character encoding angegeben werden?
Welche Möglichkeit bietet Apache in bezug auf character encoding? Wo haben Sie diese Information gefunden?}}

Prinzipiell können als Character-encoding alle von der IANA definierten
Charsets verwendet werden.(ASCII bis UTF-8)\\
https://www.iana.org/assignments/character-sets/character-sets.xhtml\\
Das zu verwendende Charset kann im HTML head angegeben werden:\\
$\langle$meta charset=''utf-8''$\rangle$\\
Diese einstellung kann jedoch von Apache überschrieben werden:\\
\textit{AddDefaultCharset}\\
Werte: Off (verwendet charset aus meta tag, abhängig von Bowser),
On (iso-8859-1), anderes IANA Charset



\section*{Aufgabe 6}
\textbf{\textit{Erklären und demonstrieren Sie die Schritte die notwendig sind, damit Sie sich ohne
Passworteingabe auf ihrem server einloggen können (Stichwort: SSH public key authen-
tication).}}

\begin{itemize}
	\item SSH Keypair auf lokalem computer erstellen:\\
	ssh-keygen -t rsa
	\item Speicherort wählen:\\
	Enter file in which to save the key (/home/.ssh/id\_rsa):
	\item Passphrase (optional):\\
	Enter passphrase (empty for no passphrase):
	\item Public Key auf Server transferieren:
	ssh-copy-id user@server ODER \\
	copy-paste.
	\item OPTIONAL: Login nur mit SSH key:\\
	sudo nano /etc/ssh/sshd\_config\\
	PermitRootLogin without-password\\
	sudo systemctl reload sshd.service
\end{itemize}

\section*{Aufgabe 7}{\tiny }
\textbf{\textit{Überprüfen Sie mittels des W3C validators (http://validator.w3.org), ob die Start-
seite Ihres servers gültiges HTML5 ist. Welche Änderungen haben Sie durchgeführt?}}

Wichtig zu berücksichtgen:
\begin{itemize}
	\item $\langle$!DOCTYPE html$\rangle$
	\item $\langle$meta charset=''utf-8''$\rangle$
\end{itemize}

\section*{Aufgabe 8}
\textbf{\textit{Wo befinden sich die log files des Webservers? Welche Informationen finden sich im log file? Was ist das common log format und wie ist es aufgebaut?}}

Im log file stehenn Informationen zu den Zugriffen auf den Webserver drinnen. Was passierte wann bei wem?
\begin{itemize}
	\item common log format:
	Standartisierter Aufbau von log files. Damit können auch Programme die files lesen und analysieren.
	
	host ident authuser date request status bytes
\end{itemize}

\end{document}
