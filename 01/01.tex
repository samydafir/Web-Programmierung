\documentclass[12pt, a4paper]{report}
\usepackage[utf8]{inputenc}
\usepackage[english]{babel}
\usepackage[backend=biber,style=numeric]{biblatex}
\usepackage{csquotes}
\usepackage{graphicx}
\usepackage{epstopdf}
\usepackage{float}
\usepackage{moreverb}

\title{Übungsblatt 01}
\author{Thomas Samy Dafir, Lex Winandy}
\date{}
\hfuzz=10.0pt

\begin{document}
\maketitle

\section{Aufgabe 3}
\begin{itemize}
	\item Benötigt Apache Modul: userdir
	\item Aktivieren: sudo a2enmod userdir
	\item Root directory festlegen: UserDir public\_html.
	\item Directory ~/public\_html anlegen
	\item Testen: buntmeise.cosy.sbg.ac.at/~username
\end{itemize}

\section{Aufgabe 6}
\begin{itemize}
	\item SSH Keypair auf lokalem computer erstellen:\\
	ssh-keygen -t rsa
	\item Speicherort wählen:\\
	Enter file in which to save the key (/home/.ssh/id\_rsa):
	\item Passphrase (optional):\\
	Enter passphrase (empty for no passphrase):
	\item Public Key auf Server transferieren:
	ssh-copy-id user@server ODER \\
	copy-paste.
	\item OPTIONAL: Login nur mit SSH key:\\
	sudo nano /etc/ssh/sshd\_config\\
	PermitRootLogin without-password\\
	sudo systemctl reload sshd.service
\end{itemize}

\section{Aufgabe 7}
Wichtig zu berücksichtgen:
\begin{itemize}
	\item $\langle$!DOCTYPE html$\rangle$
	\item $\langle$meta charset=''utf-8''$\rangle$
\end{itemize}
\end{document}





















